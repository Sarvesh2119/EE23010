\documentclass[journal,12pt,twocolumn]{IEEEtran}
\usepackage{cite}
\usepackage{amsmath,amssymb,amsfonts,amsthm}
\usepackage{algorithmic}
\usepackage{graphicx}
\usepackage{textcomp}
\usepackage{xcolor}
\usepackage{txfonts}
\usepackage{listings}
\usepackage{enumitem}
\usepackage{mathtools}
\usepackage{gensymb}
\usepackage[breaklinks=true]{hyperref}
\usepackage{tkz-euclide} % loads  TikZ and tkz-base
\usepackage{listings}
\usepackage{gvv}

\begin{document}
\bibliographystyle{IEEEtran}


\vspace{3cm}

\textbf{Question 1.5.2}
\begin{flushleft}
Find the intersection $\vec{I}$ of the angle bisectors of $B$ and $C$
\end{flushleft}

\begin{flushleft}
\textbf{Solution}\\
From (1.5.1) the bisectors of $B$ and $C$ are obtained as 
\begin{align}
\myvec{
\frac{11}{\sqrt{122}}+\frac{7}{\sqrt{74}} & \frac{1}{\sqrt{122}}+\frac{5}{\sqrt{74}}\\
}
\myvec{
x\\
y\\
}
=\frac{2}{\sqrt{74}}-\frac{38}{\sqrt{122}}
\end{align}

and 
\begin{align}
\myvec{
\frac{11}{\sqrt{122}}+\frac{1}{\sqrt{2}} & \frac{1}{\sqrt{122}}-\frac{1}{\sqrt{2}}\\
}
\myvec{
x\\
y\\
}
=\frac{2}{\sqrt{2}}-\frac{38}{\sqrt{122}}
\end{align}
respectively.\\
The pair of linear equations can be written in the form 
\begin{align}
AX &=B\\
X&=A^{-1}B
\end{align}
Here,
\begin{align}
A&=\begin{bmatrix}
\frac{11}{\sqrt{122}}+\frac{7}{\sqrt{74}} & \frac{1}{\sqrt{122}}+\frac{5}{\sqrt{74}}\\
\frac{11}{\sqrt{122}}+\frac{1}{\sqrt{2}} & \frac{1}{\sqrt{122}}-\frac{1}{\sqrt{2}}\\
\end{bmatrix}\\
B&=\begin{bmatrix}
\frac{2}{\sqrt{74}}-\frac{38}{\sqrt{122}}\\
\frac{2}{\sqrt{2}}-\frac{38}{\sqrt{122}}\\
\end{bmatrix}
\end{align}

We obtain $A^{-1}$ as
$$
\frac{1}{(\frac{6}{\sqrt{61}}+\frac{24}{\sqrt{37}\sqrt{61}}+\frac{6}{\sqrt{37}})}
\begin{bmatrix}
\frac{1}{\sqrt{2}}-\frac{1}{\sqrt{122}} & \frac{1}{\sqrt{122}}+\frac{5}{\sqrt{74}}\\
\frac{11}{\sqrt{122}}+\frac{1}{\sqrt{2}} & \frac{-11}{\sqrt{122}}-\frac{7}{\sqrt{74}}
\end{bmatrix}
$$
and $A^{-1}B$ as
$$
\frac{1}{(\frac{6}{\sqrt{61}}+\frac{24}{\sqrt{37}\sqrt{61}}+\frac{6}{\sqrt{37}})}
\begin{bmatrix}
\frac{6}{\sqrt{37}}-\frac{96}{\sqrt{37}\sqrt{61}}-\frac{18}{\sqrt{61}}\\
-\frac{30}{\sqrt{61}}+\frac{144}{\sqrt{37}\sqrt{61}}-\frac{6}{\sqrt{37}}\\
\end{bmatrix}
$$
on simplification we otain
\begin{align}
\vec{I}=\frac{1}{\sqrt{37}+4+\sqrt{61}}
\myvec{
\sqrt{61}-16-3\sqrt{37}\\
-\sqrt{61}+24-5\sqrt{37}\\
}
\end{align}

\end{flushleft}

\end{document}