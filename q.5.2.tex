\documentclass[11pt]{article}
\usepackage{amsmath}
\usepackage{setspace}
\begin{document}
\textbf{Question 1.5.2}
\begin{flushleft}
Find the intersection \textbf{I} of the angle bisectors of $B$ and $C$
\end{flushleft}

\begin{flushleft}
\textbf{Solution}\\
From (1.5.1) the bisectors of $B$ and $C$ are obtained as 
\[
\begin{pmatrix}
\frac{11}{\sqrt{122}}+\frac{7}{\sqrt{74}} & \frac{1}{\sqrt{122}}+\frac{5}{\sqrt{74}}\\
\end{pmatrix}
\begin{pmatrix}
x\\
y\\
\end{pmatrix}
=\frac{2}{\sqrt{74}}-\frac{38}{\sqrt{122}}
\]
and
\[
\begin{pmatrix}
\frac{11}{\sqrt{122}}+\frac{1}{\sqrt{2}} & \frac{1}{\sqrt{122}}-\frac{1}{\sqrt{2}}\\
\end{pmatrix}
\begin{pmatrix}
x\\
y\\
\end{pmatrix}
=\frac{2}{\sqrt{2}}-\frac{38}{\sqrt{122}}
\]
respectively.\\
\onehalfspacing
The pair of linear equations can be written in the form 
\[AX=B\]
then
\[X=A^{-1}B\]
Here,
\[
A=\begin{bmatrix}
\frac{11}{\sqrt{122}}+\frac{7}{\sqrt{74}} & \frac{1}{\sqrt{122}}+\frac{5}{\sqrt{74}}\\
\frac{11}{\sqrt{122}}+\frac{1}{\sqrt{2}} & \frac{1}{\sqrt{122}}-\frac{1}{\sqrt{2}}\\
\end{bmatrix}
\]
and 
\[
B=\begin{bmatrix}
\frac{2}{\sqrt{74}}-\frac{38}{\sqrt{122}}\\
\frac{2}{\sqrt{2}}-\frac{38}{\sqrt{122}}\\
\end{bmatrix}
\]
We obtain $A^{-1}$ as
\[
\frac{1}{(\frac{6}{\sqrt{61}}+\frac{24}{\sqrt{37}\sqrt{61}}+\frac{6}{\sqrt{37}})}
\begin{bmatrix}
\frac{1}{\sqrt{2}}-\frac{1}{\sqrt{122}} & \frac{1}{\sqrt{122}}+\frac{5}{\sqrt{74}}\\
\frac{11}{\sqrt{122}}+\frac{1}{\sqrt{2}} & \frac{-11}{\sqrt{122}}-\frac{7}{\sqrt{74}}
\end{bmatrix}
\]
and $A^{-1}B$ as
\[
\frac{1}{(\frac{6}{\sqrt{61}}+\frac{24}{\sqrt{37}\sqrt{61}}+\frac{6}{\sqrt{37}})}
\begin{bmatrix}
\frac{6}{\sqrt{37}}-\frac{96}{\sqrt{37}\sqrt{61}}-\frac{18}{\sqrt{61}}\\
-\frac{30}{\sqrt{61}}+\frac{144}{\sqrt{37}\sqrt{61}}-\frac{6}{\sqrt{37}}\\
\end{bmatrix}
\]
on simplification we otain
\[
I=\frac{1}{\sqrt{37}+4+\sqrt{61}}
\begin{pmatrix}
\sqrt{61}-16-3\sqrt{37}\\
-\sqrt{61}+24-5\sqrt{37}\\
\end{pmatrix}
\]
\end{flushleft}

\end{document}